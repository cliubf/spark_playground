\documentclass[a4paper, 12pt]{article}
    \usepackage{mathptmx}
    \usepackage{comment}
    \usepackage{fullpage}
    \usepackage{minted}
    \usemintedstyle{borland}
    \renewcommand{\baselinestretch}{2.0}

    \begin{document}
    \linespread{1.5}

    \begin{center}
        \Huge\textbf{Local Configuration with Spark and Example Usage}\\
        \large\textit{Chengzhong LIU}\\
        \normalsize\textbf{Supervisor: Prof. WANG Yu-Hsing}\\
        \normalsize\textbf{Co-Supervisor: Dr. Tan Pin Siang}
    \end{center}

    \section*{Abstract}
    Example of a Citation\cite[p.219]{Robotics}. Here's Another Citation\cite{Flueck}
    
    \section*{Installation of Apache Zeppelin on Ubuntu}
    Apache Zeppelin is one of the most powerful Spark interactive shell and widely used for research groups
    around the globe and currently supported with numerous popular interpreters\cite{Zeppelin}. However, 
    the installation of Zeppelin on local computer is not a easy work and it is easy to lost to those who
    are new to Spark or Linux. So hereby, I would like to introduce a way to easily install Zeppelin on 
    Ubuntu system (either on dual system or virtual machine). In my case, I have a virtual machine installed
    with Ubuntu, provided by VMWare\cite{VMWare}. If you are working on the UROP project in HKUST, a 60-days
    free trial should be enough. Otherwise, you might consider using VirtualBox\cite{VirtualBox} instead.\\
    \\
    Make sure your local computer have fulfill these prerequisites:
    \begin{itemize}
        \item Ubuntu system on virtual machine / dual boot system
        \item Only paste the command after \$ sign to your shell
    \end{itemize}

    \normalsize\textit{Step 1: Install homebrew}
    \begin{itemize}
        \item Initiate your ubuntu system.
        \item Press Ctrl + Alt + T to open a terminal.
        \item Paste the following command into it.
        \mint{bash}|\$ sudo apt install linuxbrew-wrapper|
        \item Then press Enter.
        \item The installation should be complete quite shortly
    \end{itemize}


    \section*{Installation of Jupyter Notebook with Spark on Windows}
    sdlkfkj
    
    \section*{Failed Attempts}
    klsdkfj
    
    \section*{Introduction to Spark Usage (on Jupyter)}
    kdjfslkfj
    % to comment sections out, use the command \ifx and \fi. Use
    %  \ifx
    %   \begin{itemize}
    %       \item item1
    %       \item item2
    %   \end{itemize}   
    %  \fi
    
    \section*{The Pipeline of Analysis}
    lksfjklsd
    
    \section*{Model Construction with MLlib}
    sdfsdf
    
    \section*{Unresolved Problem}
    lskjdfl
    
    \section*{Acknowledgement \& Future Contact}
    %Make sure to change these
    Lab Notes, HelloWorld.ic, FooBar.ic
    %\fi %comment me out
    
    \begin{thebibliography}{9}
    \bibitem{Robotics} Fred G. Martin 
    \bibitem{Flueck}  Flueck, Alexander J. 2005.
    August 2005]. Available from World Wide Web: (http://www.ece.iit.edu/~flueck/ece100).
    \bibitem{Zeppelin} Foundation, T. A. (n.d.). Apache Zeppelin. Retrieved August 3, 2018, from https://zeppelin.apache.org/
    \bibitem{VMWare} VMware – Official Site. (2018, July 30). Retrieved August 3, 2018, from https://www.vmware.com/
    \bibitem{VirtualBox} Welcome to VirtualBox.org! (n.d.). Retrieved August 3, 2018, from https://www.virtualbox.org/
    \end{thebibliography}
    
    \end{document}